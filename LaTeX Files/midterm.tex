\documentclass{article}[12]
%\setlength{\textheight}{8.4in}
%\setlength{\topmargin}{-0.15in}
%\setlength{\oddsidemargin}{0in}
%\setlength{\evensidemargin}{0in}
%\setlength{\textwidth}{6.5in}

\setlength{\textheight}{8.5in}
\setlength{\topmargin}{-.5in}
\setlength{\oddsidemargin}{-0.25in}
\setlength{\evensidemargin}{0in}
\setlength{\textwidth}{7in}
\usepackage{amsfonts, amsmath, amsthm, amssymb,mathrsfs}
\usepackage{graphicx}
\usepackage{fancyhdr}
\usepackage{setspace}
\usepackage{xcolor}
\usepackage{mathtools}
\usepackage[]{algorithm2e}
\usepackage{algorithmicx}
\usepackage{multicol}
\usepackage{subfiles}
\usepackage{tgbonum}
\pagestyle{fancy}
\lhead{TCSS 581 - Autumn 2016}
\chead{Mid-Term}
\rhead{J. McFadden}
\headsep = 22pt 
\headheight = 15pt

%\doublespacing

% BEGIN PRE-AMBLE


% Setup equation numbering 
\numberwithin{equation}{section} 

%Equation Numbering Shortcut Commands
\newcommand{\numbch}[1]{\setcounter{section}{#1} \setcounter{equation}{0}}
%\newcommand{\numbpr}[1]{\setcounter{subsection}{#1} \setcounter{equation}{0}}
\newcommand{\numbpr}[1]{\setcounter{section}{#1} \setcounter{equation}{0}}
\newcommand{\note}{\textbf{NOTE:  }}

%Formatting shortcut commands
\newcommand{\chap}[1]{\begin{center}\begin{Large}\textbf{\underline{#1}}\end{Large}\end{center}}
\newcommand{\prob}[1]{\textbf{\underline{Problem #1):}}}
\newcommand{\sol}[1]{\textbf{\underline{Solution #1):}}}
\newcommand{\MMA}{\emph{Mathematica }}

%Text Shortcut Command
\newcommand{\s}[1]{\emph{Side #1}}

% Math shortcut commands
\newcommand{\dd}[2]{\frac{d #1}{d #2}}
\newcommand{\ddn}[3]{\frac{d^{#1} #2}{d #3^{#1}}}
%\newcommand{\dd}[2]{\frac{\textrm{d} #1}{\textrm{d} #2}}
%\newcommand{\ddn}[3]{\frac{\textrm{d}^{#1} #2}{\textrm{d} #3^{#1}}}
\newcommand{\pd}[2]{\frac{\partial #1}{\partial #2}}
\newcommand{\pdn}[3]{\frac{\partial^{#1} #2}{\partial #3^{#1}}}
\newcommand{\infint}{\int_{-\infty}^\infty}
\newcommand{\infiint}{\iint_{-\infty}^\infty}
\newcommand{\infiiint}{\iiint_{-\infty}^\infty}
\newcommand{\dint}[2]{\int_{#1}^{#2}}
\newcommand{\intdd}[1]{\textrm{d}#1}
\newcommand{\intddd}[1]{\textrm{d}#1}
\newcommand{\R}{\mathbb{R}}
\newcommand{\N}{\mathbb{N}}
\newcommand{\Z}{\mathbb{Z}}
%\newcommand{\mat}[1]{\overleftrightarrow{\mathbf{#1}}}
%\newcommand{\mat}[1]{\bar{\bar{\mathbf{#1}}}}
\newcommand{\mat}[1]{\overline{\overline{\mathbf{#1}}}}

%Math Text
\newcommand{\rect}{\text{ rect}}
\newcommand{\csch}{\text{ csch}}

%Physics Shortcut Commands
\newcommand{\h}{\mathcal{H}}


%MRI Stuff Shortcut Commands
\newcommand{\tno}{t_{n}}
\newcommand{\tn}[1]{t_{n#1}}
\newcommand{\Mno}{\vec{M}^{\left( n \right)}}
\newcommand{\Mn}[1]{\vec{M}^{\left( n #1 \right)}}
\newcommand{\Mnto}[1]{\vec{M}^{(n)} \left( t_{n} #1 \right)}
\newcommand{\Mnt}[2]{\vec{M}^{(n #1)} \left( t_{n #1} #2 \right)}
\newcommand{\rot}[2]{\mat{R}_{#1} \left( #2 \right)}
\newcommand{\DnMat}[2]{\mat{D} \left( t_{n #1} #2 \right)}
\newcommand{\rotINV}[2]{\mat{R}^{-1}_{#1} \left( #2 \right)}
\newcommand{\DnMatINV}[2]{\mat{D}^{-1} \left( t_{n #1} #2 \right)}
\newcommand{\betaNt}[2]{\beta \left( t_{n #1} #2 \right)}
\newcommand{\TR}{\textrm{TR}}


% Math formatting commands
\newcommand{\stack}[2]{\stackrel{\mathclap{\normalfont\mbox{#1}}}{#2}}


% 


% END PRE-AMBLE



\begin{document}

\begin{flushleft}


\subfile{subfiles/prob1}


\vspace{0.25 in}


\subfile{subfiles/prob2}


\vspace{0.25in}


\numbpr{3}
\prob{3} One-time pads are difficult to use in practice because the any key must be the same length as the message and each key can be used only once. \newline


\vspace{0.25in}


\subfile{subfiles/prob4}


\vspace{0.25in}


\subfile{subfiles/prob5}


\vspace{0.25in}


\numbpr{6} 
\prob{6} Since any \textbf{CPA} secure encryption scheme requires than any message $m \in \mathcal{M}$ can be encrypted into any $c \in \mathcal{C}$ with some non-zero probability, any cipher-text $c$ can correspond to the encryption of several different messages, thereby preventing an eavesdropper from learning anything about plain-text by comparing two cipher-texts for equivalence. \newline


\vspace{0.25in}


\numbpr{7}
\prob{7} Each block in a cipher-text from an encryption, under \textbf{CBC} mode of operation, is dependent on either the cipher-text from the block before it or the value of \text{{\fontfamily{cmtt}\selectfont \textbf{IV}}}, they must be encrypted or decrypted serially.  Therefore, under the \textbf{CBC} mode of operation, there is no speed increase, in either encryption and decryption, available from parallel processing. \newline


\vspace{0.25in}


\subfile{subfiles/prob8}


\vspace{0.25in}


\numbpr{9}
\prob{9}  






\newpage








\numbpr{10}
\prob{10}  This scheme does not provide authentication.  In any \textbf{CPA} secure scheme, it is required that multiple, different messages can be encrypted into any arbitrary cipher-text with some non-zero probability.  Thus, we may have that $\text{{\fontfamily{cmtt}\selectfont \textbf{Enc}}}_k \left( m_1 \right) = \text{{\fontfamily{cmtt}\selectfont \textbf{Enc}}}_k \left( m_2 \right) = c$ even if $m_1 \neq m_2$.  In this case $t_1 = t_2$ would be true.  This is due to the fact that the hash function in this scheme is not keyed, therefore any arbitrary cipher-text $c$ always yields the same tag $t = \text{\textbf{hash}} \left( c \right)$.  Therefore, this scheme only ensures that the cipher-text has not been tampered with, not that it is authentic.  It does nothing to ensure that the underlying message is being represented by the received cipher-text or the cipher-text itself are authentic. \newline


\vspace{0.25in}


\numbpr{11}
\prob{11} Define the cipher-texts $c$ and $c'$, encrypted on a one-time pad with the same key, from messages $m$ and $m'$, respectively, as $c = m \oplus k$ and $c' = m' \oplus k$.  Now, \textbf{XOR} these expressions together to obtain

\begin{align}
c \oplus c' = \left( m \oplus k \right) \oplus \left( m' \oplus k \right) \notag
\end{align}

Since $k \oplus k = 0$, our result simplifies to

\begin{align}
c \oplus c' = m \oplus m' \notag
\end{align}

thereby giving an adversary a relation between cipher and message texts. \newline


\vspace{0.25in}


\numbpr{12}
\prob{12} Encryption keys can be securely reused with a stream cipher through the use of random initialization vectors and augmented pseudo random functions, $G \left( IV, k \right)$ which accept both and initialization vector and a seed and remain pseudo random even when the $IV$ is known.  Is passed at the beginning of the current session so that {\fontfamily{cmtt}\selectfont \textbf{Enc}} is defined as

\begin{align*}
{\fontfamily{cmtt}\selectfont \textbf{Enc}} := \left\langle IV, G \left( k, IV \right) \oplus m \right\rangle
\end{align*}

Additionally, we have 


\begin{align*}
{\fontfamily{cmtt}\selectfont \textbf{Dec}} := G \left( k, IV \right) \oplus c
\end{align*}

as the new definition for {\fontfamily{cmtt}\selectfont \textbf{Dec}}. \newline


\vspace{0.25in}


\numbpr{13}
\prob{13}


















































\end{flushleft}
\end{document}