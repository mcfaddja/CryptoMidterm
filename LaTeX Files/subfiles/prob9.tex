\documentclass[../midterm.tex]{subfiles}

\begin{document}
\begin{flushleft}




\numbpr{9}
\prob{9} We can calculate the probability of finding a collision here.  There are $N_{TOT} = \prod_{i = 1}^l \left\{ 2^n \right\} = \left( 2^n \right)^l$ total possible messages with $l$ blocks and block-length $n$.  Out of these, there are $N_{NO-COL} = \prod_{i = 1}^l \left\{ 2^n - i + 1 \right\}$ messages that will have no collisions because one message is removed from the number available for each subsequent block. Therefore, the number of messages with collisions in the \emph{must} be

\begin{align}
N_{COL} = N_{TOT} - N_{NO-COL} &= \prod_{i = 1}^l \left\{ 2^n \right\} - \prod_{i = 1}^l \left\{ 2^n - i + 1 \right\} \notag \\
&= \left( 2^n \right)^l - \prod_{i = 1}^l \left\{ 2^n - i + 1 \right\}
\end{align}


This allows the probability  




















\end{flushleft}
\end{document}