\documentclass[../midterm.tex]{subfiles}

\begin{document}
\begin{flushleft}




\numbpr{8}
\prob{8}  This \textbf{MAC} is \emph{not} secure because it \textbf{XOR}s the 'tags' for each message block instead of \textbf{XOR}ing all of the blocks together and then generating a tag from that result.  This is a problem because it does not prevent an adversary from changing the order of any of the blocks.   To see this, consider any two messages $m, m^\star \in \mathcal{M}$ such that $m = \left\{ m_1, m_2, \dots, m_k, \dots, m_l \right\}$, $m^\star = \left\{ m_1^\star, m_2^\star, \dots, m_k^\star, \dots, m_l^\star \right\}$, $m_i = m^\star_j$, $m_j = m^\star_i$, and $m_k = m^\star_k$ for all other $k < l$.  Then, clearly we have

\begin{align*}
t = F_k \left( m_1 \right) \oplus F_k \left( m_2 \right) \oplus \cdots F_k \left( m_i \right) \oplus \cdots \oplus F_k \left( m_j \right) \oplus \cdots \oplus F_k \left( m_l \right)
\end{align*}

which, by the the of \textbf{XOR}, is equivalent to 

\begin{align*}
t = F_k \left( m_1 \right) \oplus F_k \left( m_2 \right) \oplus \cdots F_k \left( m_j \right) \oplus \cdots \oplus F_k \left( m_i \right) \oplus \cdots \oplus F_k \left( m_l \right)
\end{align*}

Applying our definitions for $m$ and $m^\star$ from above, we clearly see that

\begin{align*}
t &= F_k \left( m_1 \right) \oplus F_k \left( m_2 \right) \oplus \cdots F_k \left( m_j \right) \oplus \cdots \oplus F_k \left( m_i \right) \oplus \cdots \oplus F_k \left( m_l \right) \\
&= F_k \left( m^\star_1 \right) \oplus F_k \left( m^\star_2 \right) \oplus \cdots F_k \left( m^\star_i \right) \oplus \cdots \oplus F_k \left( m^\star_j \right) \oplus \cdots \oplus F_k \left( m^\star_l \right) = t^\star
\end{align*}

thereby showing that an adversary can indeed change the message order with out altering the message tag.



\end{flushleft}
\end{document}