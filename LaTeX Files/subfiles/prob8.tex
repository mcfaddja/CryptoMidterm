\documentclass[../midterm.tex]{subfiles}

\begin{document}
\begin{flushleft}




\numbpr{8}
\prob{8}  This \textbf{MAC} is \emph{not} secure due to the high probability of collisions.  To see this, consider any two messages $m, m^\star \in \mathcal{M}$ such that $m = \left\{ m_1, m_2, \dots, m_k, \dots, m_l \right\}$, $m^\star = \left\{ m_1^\star, m_2^\star, \dots, m_k^\star, \dots, m_l^\star \right\}$, $m_i = m^\star_j$, $m_j = m^\star_i$, and $m_k = m^\star_k$ for all other $k < l$.  Then, clearly we have

\begin{align*}
t = F_k \left( m_1 \right) \oplus F_k \left( m_2 \right) \oplus \cdots F_k \left( m_i \right) \oplus \cdots \oplus F_k \left( m_j \right) \oplus \cdots \oplus F_k \left( m_l \right)
\end{align*}

which, by the the of \textbf{XOR}, is equivalent to 

\begin{align*}
t = F_k \left( m_1 \right) \oplus F_k \left( m_2 \right) \oplus \cdots F_k \left( m_j \right) \oplus \cdots \oplus F_k \left( m_i \right) \oplus \cdots \oplus F_k \left( m_l \right)
\end{align*}

Applying our definitions for $m$ and $m^\star$ from above, we clearly see that

\begin{align*}
t &= F_k \left( m_1 \right) \oplus F_k \left( m_2 \right) \oplus \cdots F_k \left( m_j \right) \oplus \cdots \oplus F_k \left( m_i \right) \oplus \cdots \oplus F_k \left( m_l \right) \\
&= F_k \left( m^\star_1 \right) \oplus F_k \left( m^\star_2 \right) \oplus \cdots F_k \left( m^\star_i \right) \oplus \cdots \oplus F_k \left( m^\star_j \right) \oplus \cdots \oplus F_k \left( m^\star_l \right) = t^\star
\end{align*}

thereby showing collisions for this hash function. \newline

We can calculate the probability of finding a collision here.  There are $N_{TOT} = \prod_{i = 1}^l \left\{ 2^n \right\} = \left( 2^n \right)^l$ total possible messages with $l$ blocks and block-length $n$.  Out of these, there are $N_{NO-COL} = \prod_{i = 1}^l \left\{ 2^n - i + 1 \right\}$ messages that will have no collisions because one message is removed from the number available for each subsequent block. Therefore, the number of messages with collisions in the \emph{must} be

\begin{align}
N_{COL} = N_{TOT} - N_{NO-COL} &= \prod_{i = 1}^l \left\{ 2^n \right\} - \prod_{i = 1}^l \left\{ 2^n - i + 1 \right\} \notag \\
&= \left( 2^n \right)^l - \prod_{i = 1}^l \left\{ 2^n - i + 1 \right\}
\end{align}


This allows the probability  




\end{flushleft}
\end{document}