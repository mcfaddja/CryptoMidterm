\documentclass[../midterm.tex]{subfiles}

\begin{document}
\begin{flushleft}

\numbpr{1}
\prob{1} We know that the expression

\begin{align}
\Pr \left[ M = m \, \vert \, C = c \right] = \Pr \left[ M = m \right], \label{eq1}
\end{align}

which holds for some encryption scheme {\fontfamily{cmtt}\selectfont \textbf{(Gen, Enc, Dec)}}. We apply Bayes' formula to the left-hand-side of expression \ref{eq1} to yield

\begin{align}
\Pr \left[ M = m \, \vert \, C = c \right] &= \frac{ \Pr \left[ M = m , C = c \right] }{ \Pr \left[ C = c \right] } \label{eq1a}
\end{align}

Since the $\Pr \left[ M = m , C = c \right]$ term in expression \ref{eq1a} represents a Joint Probability Distribution, we also have

\begin{align}
\Pr \left[ M = m , C = c \right] = \Pr \left[ M = m \right] \, \Pr \left[ C = c \, \vert \, M = m \right] \notag
\end{align}

This allows the result in expression \ref{eq1a} to be equivalently expressed as

\begin{align}
\Pr \left[ M = m \, \vert \, C = c \right] &= \frac{ \Pr \left[ M = m \right] \, \Pr \left[ C = c \, \vert \, M = m \right] }{ \Pr \left[ C = c \right] } \notag
\end{align}

which is divided by $\Pr \left[ M = m \right]$ to give

\begin{align}
\frac{\Pr \left[ M = m \, \vert \, C = c \right] }{ \Pr \left[ M = m \right]} &= \frac{\Pr \left[ C = c \, \vert \, M = m \right] }{ \Pr \left[ C = c \right]} \label{eq2}
\end{align}

By noting that expression \ref{eq1} implies

\begin{align*}
\frac{\Pr \left[ M = m \, \vert \, C = c \right] }{ \Pr \left[ M = m \right]} = 1,
\end{align*}

the result in expression \ref{eq2} becomes

\begin{align}
\frac{\Pr \left[ C = c \, \vert \, M = m \right] }{ \Pr \left[ C = c \right]} = 1 \notag
\end{align}

Multiplying this result by $\Pr \left[ C = c \right]$ gives

\begin{align}
\Pr \left[ C = c \, \vert \, M = m \right] = \Pr \left[ C = c \right],
\end{align}

thereby proving that $\Pr \left[ M = m \, \vert \, C = c \right] = \Pr \left[ M = m \right]$ implies $\Pr \left[ C = c \, \vert \, M = m \right] = \Pr \left[ C = c \right]$. \newline
$\square$ \newline

\end{flushleft}
\end{document}